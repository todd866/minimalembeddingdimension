\documentclass[11pt]{letter}
\usepackage[margin=1in]{geometry}
\usepackage{hyperref}

\signature{Ian Todd\\Sydney Medical School\\University of Sydney}
\address{Sydney Medical School\\University of Sydney\\Sydney, NSW 2006\\Australia\\itod2305@uni.sydney.edu.au}

\begin{document}

\begin{letter}{Editorial Office\\Information Geometry\\Springer}

\opening{Dear Editors,}

I am pleased to submit the manuscript entitled ``Minimal Embedding Dimension for Self-Intersection-Free Recurrent Processes on Statistical Manifolds'' for consideration in \textit{Information Geometry}.

This paper establishes a minimal dimension theorem for embedding cyclic processes on statistical manifolds without self-intersection. The main result---that $k=3$ is the critical threshold separating categorical from continuous representations---bridges classical differential topology (Whitney's embedding theorem, Takens' delay embedding) with the information-geometric framework.

The core contribution is Theorem 7, which proves that any continuous embedding of a cyclic process with monotone meta-time into $\mathbb{R}^2$ that preserves the cyclic structure necessarily produces self-intersections, whereas self-intersection-free embeddings into $\mathbb{R}^3$ always exist. Proposition 8 provides the information-geometric interpretation: self-intersections correspond to singularities where the induced Fisher metric becomes degenerate.

This work is appropriate for \textit{Information Geometry} because:
\begin{enumerate}
    \item It applies classical embedding theory to statistical manifolds equipped with the Fisher metric.
    \item The main theorem characterizes when dimensional collapse destroys statistical identifiability.
    \item The results connect to Watanabe's theory of singular statistical models.
\end{enumerate}

The manuscript has not been submitted elsewhere. All authors have approved the manuscript and agree with submission to \textit{Information Geometry}. There are no conflicts of interest to declare.

Thank you for considering this submission.

\closing{Sincerely,}

\end{letter}
\end{document}
